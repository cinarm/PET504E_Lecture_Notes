% This is Lecture Notes for PET504E Advanced Well Test Analysis
%
\documentclass{llncs}
\usepackage{llncsdoc}
\usepackage{savesym}
\savesymbol{note}
\usepackage{color} % Allow text colors
\usepackage{enumerate}
\usepackage{amsmath}
%\usepackage{fullpage}
%\usepackage[finalnew]{trackchanges}
%\usepackage[finalold]{trackchanges}
%\usepackage[footnotes]{trackchanges}
%\usepackage[inline]{trackchanges}
\usepackage[margins]{trackchanges}
%\usepackage[margins, movemargins]{trackchanges}
%\usepackage[margins, adjustmargins]{trackchanges}
\addeditor{Murat \c{C}{\i}nar}
\restoresymbol{Trck}{note}
%
\usepackage[usenames,dvipsnames]{pstricks}
\usepackage{epsfig}
\usepackage{pst-grad} % For gradients
\usepackage{pst-plot} % For axes
%\usepackage{nomencl}
%\makenomenclature
%
\begin{document}
\markboth{PET504E Advanced Well Test Analysis}{PET504E Advanced Well Test Analysis}
\thispagestyle{empty}
\begin{flushleft}
\LARGE\bfseries Lecture Notes\\[1cm]
\end{flushleft}
\rule{\textwidth}{1pt}
\vspace{2pt}
\begin{flushright}
\Huge
\begin{tabular}{@{}l}
PET504E \\
Advanced Well Test \\
Analysis\\[6pt]
{\Large January 2012}
\end{tabular}
\end{flushright}
\rule{\textwidth}{1pt}

\begin{flushleft}
\LARGE\bfseries by\\ Prof. Dr. Mustafa Onur \\[2cm]

\end{flushleft}
\vfill


\section{Introduction}
%
The term "Well Testing" is used in Petroleum Industry means the measuring of a formation's
(or reservoir's) pressure (and/or rate) response to flow from a well. The term "Well Testing"
is generally used with the term "Pressure Transient Analysis", interchangeably. It is an indirect
measurement technique as opposed to direct methods such as fluid sampling or coring. Well testing
provides dynamic information on the reservoir whereas direct measurements only provide static
information, which is not sufficient for predicting the behavior of the reservoir.\\

Simply, the objective of well testing is to deduce quantitative information about the well/reservoir system
under consideration from its response to a given input. Input (or input signal) is used for perturbing one or
wells so that the output (signal) exhibiting the response of the reservoir is obtained at the perturbated well and/or
adjacent wells. In practice, the input is equivalent to controlling the well behavior \remove[Murat \c{C}{\i}nar]{and} created by changing
the flow rate or the pressure at the well (Mathematically specifying the well behavior is equivalent to specifying 
a boundary condition). A common example for creating an input signal is \add[Murat \c{C}{\i}nar]{a} build up test 
\change[Murat \c{C}{\i}nar]{in which}{where} we change the rate to zero by shutting-in the well. Reservoir response,
\remove[Murat \c{C}{\i}nar]{which is} also called output signal, to a given input is monitored by measuring the
pressure change (or rate change) at the \remove[Murat \c{C}{\i}nar]{same}  well. This process is illustrated as,

\begin{figure}[h]
\setlength{\unitlength}{0.14in} % selecting unit length
\centering % used for centering Figure
\begin{picture}(32,15) % picture environment with the size (dimensions)
% 32 length units wide, and 15 units high.
\put(3,4){\framebox(6,3){Input}}
\put(13,4){\framebox(6,3){System}}
\put(23,4){\framebox(6,3){Output}}
\put(14,3){(Unknown)}
\put(6,1){I}
\put(16,1){S}
\put(26,1){O}
\put(9,5.5){\vector(1,0){4}}
\put(19,5.5){\vector(1,0){4}}
\put(9,1){\vector(1,0){4}}
\put(19,1){\vector(1,0){4}}
\end{picture}
\caption{Block diagram ?????}
\label{input_output} % label to refer figure in text
\end{figure}

Typical examples for input and output signals as used in petroleum industry are shown in the following figures;
\begin{figure}[!]
\scalebox{1} % Change this value to rescale the drawing.
{
\begin{pspicture}(0,-10.53)(15.167188,10.53)
\definecolor{color6322}{rgb}{0.6,0.6,0.6}
\psline[linewidth=0.04cm,linecolor=color6322,linestyle=dashed,dash=0.16cm 0.16cm](5.8609376,-4.79)(5.8809376,-6.19)
\pscustom[linewidth=0.04,linecolor=blue]
{
\newpath
\moveto(4.9409375,-9.45)
\lineto(5.1605444,-9.25)
\curveto(5.270349,-9.15)(5.5997605,-8.95)(5.819368,-8.85)
\curveto(6.0389757,-8.75)(6.42329,-8.65)(6.5879955,-8.65)
\curveto(6.7527027,-8.65)(7.054663,-8.675)(7.191919,-8.7)
\curveto(7.3291736,-8.725)(7.603683,-8.775)(7.7409377,-8.8)
}
\pscustom[linewidth=0.04,linecolor=red]
{
\newpath
\moveto(4.8409376,-4.95)
\lineto(5.0409374,-5.25)
\curveto(5.1409373,-5.4)(5.3659377,-5.625)(5.4909377,-5.7)
\curveto(5.6159377,-5.775)(5.8659377,-5.85)(5.9909377,-5.85)
\curveto(6.1159377,-5.85)(6.3659377,-5.85)(6.4909377,-5.85)
\curveto(6.6159377,-5.85)(6.8659377,-5.85)(6.9909377,-5.85)
\curveto(7.1159377,-5.85)(7.3409376,-5.85)(7.6409373,-5.85)
}
\psline[linewidth=0.04cm,linecolor=red](1.8809375,-5.03)(4.9409375,-4.99)
\rput(1.9009376,8.53){\psaxes[linewidth=0.04,labels=none,ticks=x,ticksize=0.08cm,Dx=3,Dy=3,showorigin=false]{->}(0,0)(0,0)(6,2)}
\usefont{T1}{ptm}{m}{n}
\rput(8.524062,8.54){time}
\usefont{T1}{ptm}{m}{n}
\rput(1.2646875,10.34){rate}
\psline[linewidth=0.04cm,linecolor=color6322,linestyle=dashed,dash=0.16cm 0.16cm](4.9009376,10.43)(4.9009376,8.71)
\usefont{T1}{ptm}{m}{n}
\rput(4.949844,8.1){$\Delta$t}
\psline[linewidth=0.04cm,linecolor=red](4.9209375,8.53)(7.6009374,8.53)
\usefont{T1}{ptm}{m}{n}
\rput(6.0085936,9.26){q = 0}
\pscustom[linewidth=0.04,linecolor=red]
{
\newpath
\moveto(1.8809375,10.37)
\lineto(2.4023657,10.26)
\curveto(2.6630812,10.205)(3.1845093,10.15)(3.4452233,10.15)
\curveto(3.7059374,10.15)(4.2273655,10.15)(4.488081,10.15)
\curveto(4.748795,10.15)(4.9573655,10.15)(4.8009377,10.15)
}
\usefont{T1}{ptm}{m}{n}
\rput(10.801094,10.285){input signal}
\rput(1.9009376,5.33){\psaxes[linewidth=0.04,labels=none,ticks=x,ticksize=0.08cm,Dx=3,Dy=3,showorigin=false]{->}(0,0)(0,0)(6,2)}
\usefont{T1}{ptm}{m}{n}
\rput(8.524062,5.34){time}
\usefont{T1}{ptm}{m}{n}
\rput(0.79640627,6.94){pressure}
\psline[linewidth=0.04cm,linecolor=color6322,linestyle=dashed,dash=0.16cm 0.16cm](4.9009376,7.23)(4.9009376,5.51)
\usefont{T1}{ptm}{m}{n}
\rput(4.949844,4.9){$\Delta$t}
\usefont{T1}{ptm}{m}{n}
\rput(10.9201565,7.085){output signal}
\usefont{T1}{ptm}{m}{n}
\rput(0.789375,6.46){at the}
\usefont{T1}{ptm}{m}{n}
\rput(0.85625,6.06){bottom hole}
\psline[linewidth=0.04cm,linecolor=blue](1.9409375,7.01)(4.8809376,5.95)
\psdots[dotsize=0.12](4.9009376,5.93)
\psline[linewidth=0.03cm,linecolor=color6322,linestyle=dashed,dash=0.16cm 0.16cm](1.8809375,7.03)(7.6209373,7.03)
\pscustom[linewidth=0.04,linecolor=blue]
{
\newpath
\moveto(4.9209375,5.93)
\lineto(5.1709375,6.242)
\curveto(5.2959375,6.398)(5.6709375,6.658)(5.9209375,6.762)
\curveto(6.1709375,6.866)(6.7334375,6.97)(7.0459375,6.97)
\curveto(7.3584375,6.97)(7.6084375,6.97)(7.4209375,6.97)
}
\rput(1.9209375,0.83){\psaxes[linewidth=0.04,labels=none,ticks=x,ticksize=0.08cm,Dx=3,Dy=3,showorigin=false]{->}(0,0)(0,0)(6,2)}
\usefont{T1}{ptm}{m}{n}
\rput(8.544063,0.84){time}
\usefont{T1}{ptm}{m}{n}
\rput(1.2846875,2.64){rate}
\psline[linewidth=0.04cm,linecolor=color6322,linestyle=dashed,dash=0.16cm 0.16cm](4.9209375,2.73)(4.9209375,1.01)
\usefont{T1}{ptm}{m}{n}
\rput(4.9171877,0.4){t$_{1}$}
\usefont{T1}{ptm}{m}{n}
\rput(12.467656,2.585){increasing rate at some time t$_{1}$}
\rput(1.9209375,-2.37){\psaxes[linewidth=0.04,labels=none,ticks=x,ticksize=0.08cm,Dx=3,Dy=3,showorigin=false]{->}(0,0)(0,0)(6,2)}
\usefont{T1}{ptm}{m}{n}
\rput(8.544063,-2.36){time}
\usefont{T1}{ptm}{m}{n}
\rput(0.81640625,-0.76){pressure}
\psline[linewidth=0.04cm,linecolor=color6322,linestyle=dashed,dash=0.16cm 0.16cm](4.9209375,-0.47)(4.9209375,-2.19)
\usefont{T1}{ptm}{m}{n}
\rput(4.9171877,-2.8){t$_{1}$}
\usefont{T1}{ptm}{m}{n}
\rput(10.900156,-0.615){two rate test}
\usefont{T1}{ptm}{m}{n}
\rput(0.809375,-1.24){at the}
\usefont{T1}{ptm}{m}{n}
\rput(0.87625,-1.64){bottom hole}
\usefont{T1}{ptm}{m}{n}
\rput(5.9307814,3.74){perturbation}
\psline[linewidth=0.04cm,arrowsize=0.05291667cm 2.0,arrowlength=1.4,arrowinset=0.4]{->}(4.9209375,3.47)(4.9209375,3.07)
\pscustom[linewidth=0.04,linecolor=red]
{
\newpath
\moveto(1.9609375,1.83)
\lineto(2.7686303,1.7340002)
\curveto(3.1724756,1.6859998)(3.8070922,1.6259998)(4.0378613,1.6140002)
\curveto(4.2686305,1.6020001)(4.614784,1.59)(4.9609375,1.59)
}
\pscustom[linewidth=0.04,linecolor=red]
{
\newpath
\moveto(4.9609375,2.43)
\lineto(5.6717067,2.3340003)
\curveto(6.027091,2.2859998)(6.585553,2.2259998)(6.7886305,2.2140002)
\curveto(6.9917064,2.2020001)(7.2963233,2.19)(7.6009374,2.19)
}
\psdots[dotsize=0.12](4.9409375,1.59)
\psdots[dotsize=0.12](4.9409375,2.45)
\pscustom[linewidth=0.04,linecolor=blue]
{
\newpath
\moveto(1.9609375,-0.93)
\lineto(2.490349,-1.02)
\curveto(2.755055,-1.065)(3.240349,-1.1325)(3.4609375,-1.155)
\curveto(3.681526,-1.1775)(4.078585,-1.2225)(4.2550564,-1.245)
\curveto(4.4315257,-1.2675)(4.696233,-1.29)(4.9609375,-1.29)
}
\pscustom[linewidth=0.04,linecolor=blue]
{
\newpath
\moveto(4.9409375,-1.29)
\lineto(5.138081,-1.41)
\curveto(5.236653,-1.47)(5.532366,-1.59)(5.7295094,-1.65)
\curveto(5.926653,-1.71)(6.288081,-1.8)(6.452366,-1.83)
\curveto(6.6166515,-1.86)(6.9452233,-1.92)(7.1095095,-1.95)
\curveto(7.273794,-1.98)(7.5695095,-2.04)(7.7009373,-2.07)
}
\psdots[dotsize=0.12](4.9209375,-1.29)
\rput(1.9009376,-6.63){\psaxes[linewidth=0.04,labels=none,ticks=x,ticksize=0.08cm,Dx=3,Dy=3,showorigin=false]{->}(0,0)(0,0)(6,2)}
\usefont{T1}{ptm}{m}{n}
\rput(8.524062,-6.62){time}
\usefont{T1}{ptm}{m}{n}
\rput(1.2646875,-4.82){rate}
\psline[linewidth=0.04cm,linecolor=color6322,linestyle=dashed,dash=0.16cm 0.16cm](4.9009376,-4.73)(4.9009376,-6.45)
\usefont{T1}{ptm}{m}{n}
\rput(4.8971877,-7.06){t$_{1}$}
\usefont{T1}{ptm}{m}{n}
\rput(10.464531,-4.875){q$_{1}$ > q$_{2}$}
\rput(1.9009376,-9.83){\psaxes[linewidth=0.04,labels=none,ticks=x,ticksize=0.08cm,Dx=3,Dy=3,showorigin=false]{->}(0,0)(0,0)(6,2)}
\usefont{T1}{ptm}{m}{n}
\rput(8.524062,-9.82){time}
\usefont{T1}{ptm}{m}{n}
\rput(0.79640627,-8.22){pressure}
\psline[linewidth=0.04cm,linecolor=color6322,linestyle=dashed,dash=0.16cm 0.16cm](4.9009376,-7.93)(4.9009376,-9.65)
\usefont{T1}{ptm}{m}{n}
\rput(4.8971877,-10.26){t$_{1}$}
\usefont{T1}{ptm}{m}{n}
\rput(10.9201565,-8.075){output signal}
\usefont{T1}{ptm}{m}{n}
\rput(0.789375,-8.7){at the}
\usefont{T1}{ptm}{m}{n}
\rput(0.85625,-9.1){bottom hole}
\usefont{T1}{ptm}{m}{n}
\rput(7.5648437,-4.14){lag due to stabilization}
\psline[linewidth=0.04cm,arrowsize=0.05291667cm 2.0,arrowlength=1.4,arrowinset=0.4]{->}(5.7409377,-4.23)(5.3809376,-4.63)
\psdots[dotsize=0.12](4.9209375,-5.01)
\usefont{T1}{ptm}{m}{n}
\rput(7.1110935,-0.48){output signal}
\psline[linewidth=0.04cm,arrowsize=0.05291667cm 2.0,arrowlength=1.4,arrowinset=0.4]{->}(5.5209374,-1.03)(6.0609374,-0.65)
\pscustom[linewidth=0.04,linecolor=blue]
{
\newpath
\moveto(1.9409375,-8.25)
\lineto(2.2909374,-8.55)
\curveto(2.4659376,-8.7)(2.9159374,-8.97)(3.1909375,-9.09)
\curveto(3.4659376,-9.21)(3.9409375,-9.36)(4.1409373,-9.39)
\curveto(4.3409376,-9.42)(4.6409373,-9.45)(4.9409375,-9.45)
}
\psline[linewidth=0.04cm,linecolor=color6322,linestyle=dashed,dash=0.16cm 0.16cm](5.8609376,-7.99)(5.8809376,-9.51)
\psdots[dotsize=0.12](4.9009376,-9.47)
\usefont{T1}{ptm}{m}{n}
\rput(3.2929688,-4.54){q$_{1}$}
\usefont{T1}{ptm}{m}{n}
\rput(6.9075,-5.34){q$_{2}$}
\end{pspicture}
}
\caption{Typical input and output signals - Transient phenomena.}
\label{Transient_phenomena} 
\end{figure}


From reservoir response as monitored by the "output signal", we would like to determine information related to the followings
\begin{itemize}
    \item Fluid in place; pore volume, ${\phi}hA$.
    \item Ability of reservoir to transfer fluid, $kh$ (or transmissibility, $\frac{kh}{\mu}$).
    \item Determination of average reservoir \change[Murat \c{C}{\i}nar]{average}{pressure}, $\overline{P}$, which is the driving force in the reservoir \Trcknote[Murat \c{C}{\i}nar]{Based upon the explanation you gave about the pressure decline recently, I am not sure if this statement is correct.}
    \item Prediction of rate versus time data.
    \item Initial recovery, is the reservoir worth producing.
    \item Is there any damage around the wellbore impeding the flow? skin factor, $s$.
    \item Reservoir description (type of reservoir, flow boundaries (faults)).
    \item Distance to fluid interface \change[Murat \c{C}{\i}nar]{which}{that} is important determining swept zone for secondary and tertiary methods.
\end{itemize}
Interpretation of well test data consist of basically three steps:

\begin{enumerate}[(i)]
\item Determination of the one most appropriate reservoir / wellbore (mathematical) model \change[Murat \c{C}{\i}nar]{to}{of} the actual system. We also call such a model as the interpretation model. \change[Murat \c{C}{\i}nar]{Here our hope is that the model chosen will produce an output signal to a given input which is as close as possible to that of the actual system.}{Here our intention is to find a representative mathematical model that reproduces, as close as possible, the output of the actual system for a given input.} This is known as the inverse problem. \add[Murat \c{C}{\i}nar]{We are trying to obtain information about the physical system by using observed measurements.} Unfortunately, the solution of inverse problem often yields non-unique results. \change[Murat \c{C}{\i}nar]{With}{By} non-unique results, we mean that several different interpretation models \change[Murat \c{C}{\i}nar]{can}{may} generate an output signal (response) to a given input \change[Murat \c{C}{\i}nar]{which}{that} is similar (or identical) to that of the actual system. The inverse problem can be represented by the following equation.
    \begin{equation}
    \Sigma ={O}/{I}\;\approx S
    \label{Inverse_P}
    \end{equation} 
    where $\Sigma$ denotes the interpretation model, $S$ denotes the actual system. In inverse problem, as can be seen from Eq.\ref{Inverse_P}, it may be possible to obtain the same outputs to a given $I$ for different $\Sigma_{i}$'s\change[Murat \c{C}{\i}nar]{. However}{,however}, the number of alternative models (solutions) can be reduced as the number and the range of output signal measurements.
    
\item determinatio ndldf komgf. 
\end{enumerate}


%\nomenclature[1]{$R$}{Ideal gas constant {$(8.314 J/gmol \cdot K)$}}%
%\nomenclature{${\alpha }$}{Reaction order}%
%\printnomenclature
\bibliographystyle{plain}
\bibliography{References}

%
\end{document}
